\documentclass[a4paper]{article}
\usepackage[14pt]{extsizes}
\usepackage[utf8]{inputenc}
\usepackage[T1, T2A]{fontenc}
\usepackage[a4paper, top=2cm, bottom=2cm, left=3cm, right=1.5cm, marginparwidth=1.75cm, nohead, footskip=10mm]{geometry}
\usepackage{graphicx}
\usepackage{amsmath}
\usepackage{amssymb}
\usepackage{amsfonts}
\usepackage{indentfirst}
\usepackage[english, russian]{babel}
\usepackage[section,above,below]{placeins}
\usepackage{pdfpages} 
\usepackage{svg}
\usepackage[colorlinks,urlcolor=blue]{hyperref}
\usepackage{multirow}
\usepackage{flafter}
\usepackage[normalem]{ulem}
\usepackage{setspace}
\usepackage{mathdots}

\begin{document}

\noindent \textbf{СОДЕРЖАНИЕ}

\noindent Введение            4

\noindent 1 Определения           6

\noindent    1.1 Определение центра области, радиуса, подобных контуров  6

\noindent    1.2 Определение целевой и побочной задачи, неустойчивости 

\noindent    решения                 11

\noindent 2 Обратная задача гравиметрии (ОЗГ)           14

\noindent    2.1 Постановка задачи              14

\noindent    2.2 Алгоритм приближённого решения           14

\noindent         2.2.1 Описание алгоритма решения ОЗГ          15

\noindent         2.2.2 Детали реализации             16

\noindent    2.3 Примеры решения ОЗГ             25

\noindent         2.3.1 Параметры численного метода           26

\noindent         2.3.2 Аппроксимация на границе области          26

\noindent         2.3.3 Аппроксимация внутри области           32

\noindent    2.4 Устойчивость решения ОЗГ в классе гармонических 

\noindent    плотностей                37

\noindent         2.4.1 О регулярных контурах            37

\noindent         2.4.2 Вывод $V_{{\rho }_k}$               38

\noindent         2.4.3 Некорректность ОЗГ             40

\noindent    2.5 Демонстрация неустойчивости при численном решении       42

\noindent         2.5.1 Реализация решения             42

\noindent         2.5.2 Примеры расчётов             44

\noindent 3 Краевая задача для бигармонического уравнения\textit{         }50

\noindent    3.1 Постановка задачи              50

\noindent    3.2 Численное решение              50

\noindent         3.2.1 Решение сведением к ОЗГ (метод 1)          50

\noindent         3.2.2 Метод фундаментальных решений для бигармонического   

\noindent         уравнения (метод 2)              51

\noindent    3.3 Аппроксимация граничных условий и точного решения внутри

\noindent    области 2-м методом                 55

\noindent    3.4 Сравнение методов              58

\noindent Заключение                66

\noindent 


\end{document}

