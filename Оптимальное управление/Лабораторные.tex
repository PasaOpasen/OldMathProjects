\documentclass[a4paper, 12pt]{article}
\usepackage[utf8]{inputenc}
\usepackage[T1,T2A]{fontenc}
\usepackage[a4paper, top=2cm, bottom=2cm, left=1cm, right=1cm, marginparwidth=1.75cm]{geometry}
\usepackage{graphicx}
\usepackage{amsmath}
\usepackage{indentfirst}
\usepackage[english, russian]{babel}
\usepackage[section,above,below]{placeins}
\usepackage[noend]{algorithmic}
\usepackage{amssymb}
\usepackage{amsfonts}
\usepackage{pdfpages}

\newcommand{\df}[2]{\frac{\partial #1}{\partial #2}}

\begin{document}
\section{Лабораторная 1}
{\bf Дана задача}:
\begin{equation}
    \dfrac{\partial K(t)}{\partial t}=L(t)+3K(t), K(0)=2,0\le L < \infty, t \in [0,2]
\end{equation}    
\begin{equation}
    I(K,L)=\int_0^2 K(t)+L^2(t) dt - 2 K(2) \rightarrow \min.
\end{equation}

{\bf Решение}:

Оптимальный процесс является решением вспомогательной задачи
\begin{equation}
    H(K,L,p)=-(K+L^2)+p(L+3K)\rightarrow \max_L,
\end{equation}
то есть 
\begin{equation}
   \df{H}{L}=-2L+p=0 \Rightarrow L=\frac{p}{2}.
\end{equation}

Сопряженная задача имеет вид:
\begin{equation}
    \dot p = -\df{H}{K}=1-3p,
\end{equation}
\begin{equation}
    p(2)=-\df{(-2K(2))}{K(2)}=2.
\end{equation}
Дифференциальное уравнение имеет решение $p=C_1 \dfrac{e^{-3t}}{3} +\dfrac{1}{3}$, причем $p(2)=2$, поэтому $C_1=5 e^{6}$, тогда
\begin{equation}
    p(t)=\dfrac{1}{3}\left(5e^{6} e^{-3t}+1 \right) ,
\end{equation}
\begin{equation}
    L(t)=\dfrac{1}{6}\left(5e^{6} e^{-3t}+1 \right)>0,
\end{equation}
\begin{equation}
    \df{K}{t}=3K+\dfrac{5}{6} e^{-3t+6}+\dfrac{1}{6} \Rightarrow K=\left(C_1- \dfrac{c e^{-3t}}{3}- \dfrac{k e^{-6t}}{6}\right) e^{3t},c=\dfrac{1}{6},k=\dfrac{5 e^{6}}{6}.
\end{equation}

Из условия $K(0)=2$ находим $C_1=2\dfrac{1}{18}+\dfrac{5 e^{6}}{36}$, откуда 
\begin{equation}
    K(t)=K=\left(2\dfrac{1}{18}+\dfrac{5 e^{6}}{36}- \dfrac{e^{-3t}}{18}-\dfrac{5 e^{6} e^{-6t}}{36}\right) e^{3t}.
\end{equation}


\section{Лабораторная 2}

\end{document}